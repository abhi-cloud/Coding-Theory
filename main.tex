\documentclass[11pt]{article}
\usepackage{mathrsfs, amsmath, amssymb, amsthm, amsfonts}
\usepackage[inline]{enumitem}
\usepackage[none]{hyphenat}
\usepackage{graphicx}
\graphicspath{{images/}{../images}}
\usepackage{float}
\usepackage{fancyhdr}
%\usepackage[nottoc, notlot, notlof]{tocbibind}
\usepackage{tikz}

\usepackage{geometry}
\geometry{
	a4paper,
	total={170mm,257mm},
	left=20mm,
	top=20mm,
}

\usepackage{titlesec}
\titleformat{\section}[block]
  {\normalfont\scshape\Large}{\S\thesection}{0.25cm}{\Large}

\theoremstyle{definition}
\newtheorem{thm}{Theorem}[section]
\newtheorem{defn}[thm]{Definition}
\newtheorem{cor}{Corollary}[thm]
\newtheorem{lem}[thm]{Lemma}

\theoremstyle{remark}
\newtheorem*{remark}{Remark}

\renewcommand{\baselinestretch}{0.85}
\renewcommand\qedsymbol{$\square$}
\newcommand{\fl}[1]{\left\lfloor #1 \right\rfloor}
\newcommand{\Mod}[1]{\ (\mathrm{mod}\ #1)}
\newcommand{\dis}[2]{d(\textbf{#1},\textbf{#2})}
\newcommand{\code}[1]{\textbf{#1}}

\setlength\parindent{0pt}
%\setlength\parskip{0.5em plus 0.1em minus 0.2em}
\let\emptyset\varnothing

\pagestyle{fancy}
\fancyhead{}
\fancyfoot{}
\fancyhead[R]{\textsl{Coding Theory}}
\renewcommand{\sectionmark}[1]{\markboth{#1}{}}

\fancyhead[L]{\nouppercase{\leftmark}}
\fancyfoot[R]{\thepage}
\renewcommand{\footrulewidth}{0pt}

\usepackage{subfiles}

\begin{document}

\subfile{sections/titlepage}

\section*{Preface}
In this report, we will try to understand how a message/data is sent across noisy channels. Messages sent through such channels undergo some aberration due to various kinds of noises. Our motto is to have a high probability of decoding a recieved message to its original meaning. This report goes through such encoding and decoding procedures, from fairly trivial to some advanced methods that enable us with fast and accurate decoding schemes. We will be following the text \emph{A first Course in Coding Theory}\cite{mostly} throughout this report, with some help from \emph{Introoduction to Coding Theory	}\cite{rarely}.
  
\tableofcontents
\thispagestyle{empty}
\newpage

\setcounter{page}{1}

\section*{Notation\markboth{Notation}{}}
\addcontentsline{toc}{section}{\numberline{}Notation}
\subfile{sections/notation}
\newpage
\section{Introduction to error-correcting codes}
\subfile{sections/introToERCCs}
\newpage
\section{The main coding theory problem}
\subfile{sections/mainCTproblem}
\newpage
\section{Finite fields and Vector spaces over finite fields}
\subfile{sections/fieldsnVectorSpaces}
\newpage
\section{Linear Codes}
\subfile{sections/linearcodes}
\newpage
\section{Dual Code, parity-check matrix and syndrome decoding}
\subfile{sections/dualcodesNpcms}
\newpage
\section{The Hamming Codes}
\subfile{sections/hammingCodes}
\newpage
\section{Codes and Latin Squares}
\subfile{sections/latin_squares}
\newpage
\section{Basic BCH codes}
\subfile{sections/bch_codes}
\newpage
\section{Cyclic codes}
\subfile{sections/cyclic_codes}
\newpage
\section{Weight enumerators}
\subfile{sections/weight_enum}
\newpage

\iffalse
\addcontentsline{toc}{section}{\numberline{}Updated PoA}
\section*{Updated PoA}
I have mostly covered what I had planned till midterm (adding worked out problems is left), so I'll continue with the original plan i.e. trying to complete the text, \emph{A first Course in Coding Theory}\cite{mostly}, with some help from \emph{Introoduction to Coding Theory	}\cite{rarely}.

\begin{itemize}
	\item Week 4 : \textit{May 2, 2020 - May 8, 2020}
	\begin{itemize}
		\itemsep-1mm
		\item Chapter 8
		\item Chapter 9
	\end{itemize}
\end{itemize}

\begin{itemize}
	\item Week 5 : \textit{May 9, 2020 - May 15, 2020}
	\begin{itemize}
		\itemsep-1mm
		\item Chapter 9
		\item Chapter 10
	\end{itemize}
\end{itemize}

\begin{itemize}
	\item Week 6 : \textit{May 16, 2020 - May 22, 2020}
	\begin{itemize}
		\itemsep-1mm
		\item Chapter 11
		\item Chapter 12
	\end{itemize}
\end{itemize}

\begin{itemize}
	\item Week 7 : \textit{May 23, 2020 - May 29, 2020}
	\begin{itemize}
		\itemsep-1mm
		\item Chapter 13
		\item Chapter 14
	\end{itemize}
\end{itemize}
\fi

\medskip
\addcontentsline{toc}{section}{\numberline{}References}
\bibliographystyle{unsrt}
\bibliography{ref.bib}

\end{document}