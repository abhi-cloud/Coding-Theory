\documentclass[../main.tex]{subfiles}

\newcommand{\poly}[1]{$#1(x)$}
\newcommand{\cyc}[1]{\langle #1(x) \rangle}

\begin{document}

\subsection{Definition}

Cyclic codes form an important type of codes for several theoritical as well as practical reasons. From theoritical perspective, they can be expressed as a rich algebraic structure, while practically they can be efficiently implemented. Furthermore, various important codes such as binary Hamming codes and BCH codes, are equivalent to cyclic codes.\\

\begin{defn}
	A code $C$ is \textbf{\emph{cyclic}} if 
	\begin{enumerate*}[label=(\roman*), before=\unskip{}]
		\item $C$ is a linear code; and 
		\item any cyclic shift of a codeword is also a codeword, i.e. whenever $a_0a_1\cdots a_{n-1}$ is in $C$, then so is $a_{n-1}a_0a_1\cdots a_{n-2}$.
	\end{enumerate*}
\end{defn}

\textbf{Example:}\\
The linear code $\{ 0000,\; 1001,\; 0110,\; 1111 \}$ is not cyclic, but it is \emph{equivalent} to a cyclic code; interchanging the third and fourth coordinates gives the cyclic code $\{ 0000,\; 1010,\; 0101,\; 1111 \}$.\\

When considering cyclic codes we number the coordinate positions $0,1,\ldots,n-1$. This is because then a vector $a_0a_1\cdots a_{n-1}$ in $V(n,q)$ correspond to the polynomial $a_0 + a_1x + \cdots +a_{n-1}x^{n-1}$.

\subsection{Polynomials over finite fields}

We denote by $F[x]$ the set of polynomials in $x$ with coefficients in $F_q$ (or simply $F$ with $q$ understood).\\

\emph{Degree} of a polynomial in $F[x]$ is defined as usual. These polynomials can be added, subtracted and multiplied in the usual way, thus forming the algebraic structure, \emph{ring}, not a field as they do not have multiplicative inverses.\\
\emph{Division algorithm} states the same as in with the polynomials over $\mathbb{R}$, i.e. for every pair of polynomials $a(x)$ and $b(x)\neq 0$ in $F[x]$, there exists a unique polynomials $q(x)$, and $r(x)$, such that
\[
	a(x) = q(x)b(x) + r(x),
\] 
where deg \poly{r} $<$ deg \poly{b}.\\

Now we will establish similarites between the ring $F[x]$ of polynomials and the ring $\mathbb{Z}$ of integers.\\
Just as the ring $\mathbb{Z}_m$ is obtained after modulo $m$, we can consider polynomials in $F[x]$ modulo some \poly{f}. It is natural to define \poly{g} and \poly{h} as \emph{congruent modulo} \poly{f}, symbolized by
\[
	g(x) \equiv h(x) \Mod{f(x)}
\] 
if \poly{g} $-$ \poly{h} is divisible by $f(x)$.\\

We denote by $F[x]/f(x)$ the set of polynomials in $F[x]$ of degree less than deg \poly{f}, with addition defined as normal addition as adding two polynomials in $F[x]/f(x)$ will not increase the degree; while multiplication is defined congruent modulo $f(x)$, i.e for any two polynomials in $F[x]/f(x)$, a unique polynomial of degree less than deg \poly{f}.\\

Finally, the set $F[x]/f(x)$ is called a \emph{ring of polynomials(over $F$) modulo \poly{f}} and it is quite obvious that 
\[
	|F_q[x]/f(x)| = q^n.
\]

Now the next logical step is to find out when this ring forms a field, i.e. when each of the polynomials in $F[x]/f(x)$ have a multiplicative inverse.\\ 
For that we will first define \emph{reducibility} of polynomials.

\begin{defn}
	A polynomial \poly{f} is called \textbf{\emph{reducible}} if $f(x) = a(x)b(x)$, where $a(x), b(x) \in F[x]$ and deg \poly{a}, deg \poly{b} are both smaller than deg \poly{f}. If \poly{f} is not \emph{reducible}, it is called \textbf{\emph{irreducible}}. 
\end{defn}

\begin{thm}
	The ring $F[x]/f(x)$ is a field if and only if $f(x)$ is irreducible in $F[x]$. 
\end{thm}
\begin{proof}
	The proof follows the same line as followed by the proof of Theorem \ref{thm_Zm_field}, with prime $m$ being replaced by $f(x)$.
\end{proof}

\subsection{Cyclic codes expressed as polynomials}

From now on, we will fix $f(x)=x^n-1$, as the ring $F[x]/(x^n-1)$ of the polynomials modulo $x^n-1$ is the natural one to consider in the context of cyclic codes. We will now denote $F[x]/(x^n-1)$ as $R_n$, where the field $F=F_q$ will be understood.\\
Since $x^n \equiv 1 \Mod{x^n-1}$, we can reduce any polynomial modulo $x^n-1$ simply by replacing $x^k$ by $x^{k\Mod{n}}$, without any long division.\\

We will now denote a vector $a_0a_1\cdots a_{n-1}$ in $V(n,q)$ with the polynomial
\[
	a(x) = a_0 + a_1x + \cdots + a^{n-1}x^{n-1}
\]
in $R_n$, that is, now a code $C$ is subset of both $V(n,q)$ and $R_n$. Note that addition of vectors and multiplication of a vector by a scalar in $R_n$ corresponds exactly to those operations in $V(n,q)$.\\
Now, it must be clear that \emph{multiplying} by $x^m$ to \poly{a} corresponds to a cyclic shift through $m$ positions.\\
We can model cyclic codes in a way, given by the following theorem.

\begin{thm}\label{thm_alg_char}
	A code $C$ in $R_n$ is a cyclic code if and only if $C$ satisfies the following two conditions: 
	\begin{enumerate}[label=(\roman*)]
		\item \poly{a}, \poly{b} $\in$ $C \implies$ \poly{a} + \poly{b} $\in C$,
		\item \poly{a} $\in C$ and \poly{r} $\in R_n \implies$ \poly{r}\poly{a} $\in C$. 
	\end{enumerate}
\end{thm}

\begin{proof}
	Suppose $C$ is a cyclic code in $R_n$. Then $C$ is linear and so $(i)$ holds. Now suppose $a(x) \in C$ and $r(x) = r_0 + r_1x + \cdots + r_{n-1}x^{n-1} \in R_n$. Since, $x^m a(x) \in C \forall m$ (cyclic shifts). Hence,
	\[
		r(x)a(x) = r_0a(x) + r_1xa(x) + \cdots + r_{n-1}x^{n-1}a(x)
	\]
	is also in $C$ since each summand is in $C$. Thus, $(ii)$ also holds.\\
	Now suppose $(i)$ and $(ii)$ hold. Taking $r(x)$ as a scalar, the conditions imply that $C$ is linear. Taking $r(x)=x$ in $(ii)$ shows that $C$ is cyclic.
\end{proof}

Now we have a easy way of constructing cyclic codes. Let $f(x)$ be any polynomial in $R_n$, then we define
\[
	\langle f(x) \rangle = \{ r(x)f(x) | r(x) \in R_n \}
\]
$\cyc{f}$ is a cyclic code for all $f(x)$ in $R_n$, as it satisfies the conditions of Theorem \ref{thm_alg_char}.

\begin{thm}\label{thm_existance_genpoly}
	Let $C$ be a non-zero cyclic code in $R_n$. Then
	\begin{enumerate}[label=(\roman*)]
		\item there exists a unique monic (coefficient of highest degree term $1$) polynomial $g(x)$ of smallest degree in $C$,
		\item $C = \cyc{g}$,
		\item $g(x)$ is a factor of $x^n-1$.\\
	\end{enumerate}
\end{thm}

\begin{proof}
	\begin{enumerate*}[label=(\roman*), before=\unskip{}]
	\item Suppose $g(x)$ and $h(x)$ are both monic polynomials in $C$ of the smallest degree. Then $g(x)-h(x) \in C$ and has smaller degree. This gives a contradiction if $g(x) \neq h(x)$, for then a suitable scalar multiple of $g(x)-h(x)$ is monic.\\
	
	\item Suppose $a(x) \in C$. By the division algorithm for $F[x]$, $a(x)=q(x)g(x) + r(x)$, where deg \poly{r} $<$ deg \poly{g}. But $r(x)$, belongs to $C$, as $C$ is linear. By minimality of deg \poly{g}, we must have $r(x)=0$ and so $a(x) \in \cyc{g}$.\\
	
	\item Again by division algorithm, 
	$
		x^n - 1 = q(x)g(x) + r(x) 
	$ 
	where deg \poly{r} $<$ \poly{g}. But then $r(x) \equiv -q(x)g(x) \Mod{x^n-1}$, and so $r(x) \in \cyc{g}$. By minimality again, $r(x)=0$, which implies $g(x)$ is a factor of $x^n-1$.	
	\end{enumerate*}
\end{proof} 

\begin{defn}
	In a non-zero cyclic code $C$ the monic polynomial of least degree, given by theorem \ref{thm_existance_genpoly}, is called the \textbf{\emph{generator polynomial}} of $C$.
\end{defn}

The third part of Theorem \ref{thm_existance_genpoly} gives us method of finding all the cyclic codes of length $n$. All we need are the factors of $x^n-1$.\\

\textbf{Example:}\\
Finding all the binary cyclic codes of length $3$. We have $x^3-1 = (x+1)(x^2+x+1)$, where $x+1$ and $x^2+x+1$ are irreducible over $GF(2)$. So the codes are
\begin{center}
\begin{tabular}{|c|c|c|}
\hline
Generator ploynomial & Code in $R_3$ & Corresponding code in $V(3,2)$ \\
\hline
$1$ & all of $R_3$ & all of $V(3,2)$ \\
$x+1$ & $\{ 0, 1+x, x + x^2, 1 + x^2 \}$ & $\{ 000, 110, 011, 101 \}$ \\
$x^2+x+1$ & $\{ 0, 1 + x+ x^2 \}$ & $\{ 000, 111 \}$ \\
$x^3-1 = 0$ & $\{ 0 \}$ & $\{ 000 \}$ \\
\hline
\end{tabular}
\end{center}

\begin{lem}
	Let $g(x) = g_0 + g_1x + \cdots + g_rx^r$ be the generator polynomial of a cyclic code, then $g_0$ is non-zero.
\end{lem}

\begin{proof}
	Suppose $g_0 = 0$. Then $x^{n-1}g(x) = x^{-1}g(x)$ is a codeword of $C$ of degree $r-1$, contradicting the minimality of deg \poly{g}.
\end{proof}

Now, we are going to define generator matrix (thus, dimensions of $C$) directly from generator polynomial.

\begin{thm}
	Suppose $C$ is a cyclic code with generator polynomial
	\[
		g(x) = g_0 + g_1x + \cdots + g_rx^r
	\]
	of degree $r$. Then dim $(C)$ = $n-r$ and a generator matrix $C$ is
	\[
		G=
		\begin{bmatrix}
			g_0 & g_1 & g_2 & \cdots & & g_r & 0 & 0 & \cdots & 0 \\
			0 & g_0 & g_1 & g_2 & \cdots &  & g_r & 0 & \cdots & 0 \\
			0 & 0 & g_0 & g_1 & g_2 & \cdots &  & g_r &  & \vdots \\
			\vdots & \vdots & & \ddots & \ddots & \ddots & & & \ddots & 0 \\
			0 & 0 & \cdots & 0 & g_0 & g_1 & g_2 & \cdots &  & g_r \\
		\end{bmatrix}
	\]
\end{thm}

\begin{proof}
	The $n-r$ rows of the above matrix $G$ are certainly linearly independent due echelon of non-zero $g_0$s. The $n-r$ rows represent the cyclic permutations of codeword $g(x)$. The proof of Theorem \ref{thm_existance_genpoly}(ii) shows that if $a(x)$ is a codeword of $C$, then
	\[
		a(x) = q(x)g(x)
	\]
	for some polynomial $q(x)$, and that this is an equality of polynomials within $F[x]$, i.e. without any modulo. Since deg \poly{a} $< n$, it follows that deg $q(x) < n-r $. Hence,
	\begin{align*}
		q(x)g(x) &= (q_0 + q_1x + \cdots + q_{n-r-1}x^{n-r-1})g(x) \\
		&= q_0g(x) + q_1xg(x) + \cdots + q_{n-r-1}x^{n-r-1}g(x),
	\end{align*}
	which shows that every codeword can be written as a linear combination of those $n-r$ rows.
\end{proof}

\begin{defn}
	Let $C$ be a cyclic $[n,k]$-code with generator polynomial $g(x)$. By Theorem \ref{thm_existance_genpoly} \poly{g} is a factor of $x^n - 1$ and so
	\[
		x^n - 1 = g(x)h(x),
	\]
	for some polynomial $h(x)$. \poly{h} is of degree $k$, and is called the \textbf{\emph{check-polynomial}} of $C$.
\end{defn}

\begin{thm}\label{thm_orth}
	Suppose $C$ is a cyclic code in $R_n$ with generator polynomial \poly{g} and check polynomial \poly{h}. Then \poly{c} $\in R_n$ is a codeword in $C$ if and only if $c(x)h(x)=0$.
\end{thm}

\begin{proof}
	The forward implication is trivial as \poly{g}\poly{h} $=0$. On the other hand, suppose $c(x)$ satisfies $c(x)h(x) = 0$. If \poly{r} is the remainder of \poly{c} with \poly{g} then $r(x)h(x) = 0\Mod{x^n-1}$. But deg $(r(x)h(x)) < n - k + k = n$, so $r(x) = 0$, then $c(x) = q(x)g(x) \in C$.
\end{proof}

\begin{thm}
	Suppose $C$ is a cyclic $[n,k]$-code with check polynomial
	\[
		h(x) = h_0 + h_1x + \cdots + h_kx^k
	\]
	Then a parity check matrix for $C$ is
	\[
		H = 
		\begin{bmatrix}
			h_k & h_{k-1} & \cdots & h_0 & 0 & 0 & \cdots & 0 \\
			0 & h_k & h_{k-1} & \cdots & h_0 & 0 & \cdots & 0 \\
			& & \ddots & \ddots & & \ddots & & 0 \\
			0 & \cdots & 0 & h_k & h_{k-1} & \cdots &  & h_0 \\
		\end{bmatrix}
	\]
\end{thm}

\begin{proof}
	By Theorem \ref{thm_orth}, \poly{c}$= c_0 + c_1x + \cdots + c_{n-1}x^{n-1}$ is codeword if and only if $c(x)h(x)=0$. Thus any codeword $c_0c_1\cdots c_{n-1}$ of $C$ is orthogonal to the vector $h_kh_{k-1}\cdots h_0\cdots 0$ and to its cyclic shifts (this results from equating coefficients of $x^k, x^{k+1}, \ldots, x^{n-1}$ must all be zero in $c(x)h(x))$. Thus all the $n-k$ rows are orthogonal to all the codewords and are linearly independent (becuase echelon of $h_k = 1$ in $H$), and the dimension of $C^{\perp}$ is also $n-k$. Hence, $H$ is a generator matrix of $C^{\perp}$.  
\end{proof}

\end{document}